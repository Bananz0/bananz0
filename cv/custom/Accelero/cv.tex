\documentclass[11pt,a4paper,sans]{moderncv}
\moderncvstyle{banking}
\moderncvcolor{blue}
\renewcommand{\sfdefault}{lmss} 
\usepackage[scale=0.85]{geometry}
\usepackage{lastpage}
\usepackage{enumitem}
\usepackage{hyperref}
\usepackage{fontawesome5}
\setlist{noitemsep, leftmargin=*}

% Personal Information
\name{Glen}{Muthoka Mutinda}
\phone{+44 7341 625286}
\email{theglenmuthoka@gmail.com}
\social[linkedin]{glenmuthoka}
\social[github]{Bananz0}

\begin{document}

\makecvtitle

\section{Professional Summary}
Hands-on final-year Electrical \& Electronics Engineering student (Expected First Class Honours, June 2026) with proven ability to move fast and deliver results. Strong foundation in embedded C/C++ development, Python automation, and ARM/STM32 microcontrollers. Track record of building real-world prototypes from concept to working hardware—from space-grade CubeSat software to IoT energy systems. Thrives in fast-paced environments where experimentation and problem-solving drive innovation. Ready to shape products from the ground up in a startup setting.

\section{Education}
\cventry{2023--2026}{BEng Electrical \& Electronics Engineering (Expected First Class Honours)}{University of Southampton}{Southampton, UK}{}{
\textbf{Specialization:} Embedded Systems, IoT Architecture, PCB Design, Real-Time Systems\\
\textbf{Second Year Project:} TSMC 65nm IC Design \& Fabrication (Team Lead - managing 6 engineers)\\
\textbf{Final Year Project:} AI-Driven Optical Authentication Framework using Optical PUFs (Lead Researcher)}

\cventry{2022--2023}{Undergraduate Foundation Programme, Engineering Pathway (Distinction)}{ONCAMPUS Global}{Southampton, UK}{}{}

\section{Professional Experience}
\cventry{Sep 2023--Jun 2025}{Junior Software Engineer - Embedded Systems}{ARTEMIS Small Sat-1 Lunar CubeSat Project}{University of Southampton}{}{
\begin{itemize}
    \item Developed flight software in C/C++ for ARM Cortex-M microcontroller in resource-constrained, mission-critical environment—similar pressure to startup product development
    \item Reduced system latency by 6\% through iterative optimization and experimentation with different approaches
    \item Built custom device drivers for SPI, I2C, and UART peripherals, debugging hardware-software integration issues
    \item Created Python automation scripts for testing, data analysis, and build processes, accelerating development cycle
    \item Collaborated directly with small multidisciplinary team (8 engineers) to rapidly iterate on prototypes and meet tight deadlines
    \item Enhanced system reliability by 20\% through test-driven development and continuous improvement mindset
\end{itemize}}

\section{Technical Projects - Building Real Products}

\cvitem{WattsApp - IoT Smart Energy Management Platform}{Jan 2025--Mar 2025\\
\textbf{Full-Stack IoT Prototype} | ESP32 | C | Python | React.js | MQTT | Modbus\\
Built complete end-to-end IoT system from scratch—embedded firmware, backend infrastructure, and web dashboard. Developed bare metal C firmware for ESP32 managing power measurement sensors (INA219, ACS712), load balancing algorithms, and wireless communication. Implemented Python backend for data processing and MQTT broker management. Created React.js dashboard for real-time visualization. Achieved 12\% reduction in peak grid consumption through intelligent automation. \textbf{This is exactly the kind of hardware-software integration, prototyping, and problem-solving I thrive on.}\\
\faGithub~\href{https://github.com/Bananz0/WattsApp}{github.com/Bananz0/WattsApp}}

\cvitem{eGPU Auto-Enabler - Hot-Plug Automation Tool}{2024--Present\\
\textbf{PowerShell} | Windows API | Device Management\\
Shipped production-ready Windows automation tool solving real user problems—exactly the "make things happen" mentality. Implemented automatic device detection, custom power management, crash recovery, and auto-update system. Achieved 99.9\% reliability across diverse hardware configurations. Built user base of 8,300+ through iterative improvements based on feedback. Demonstrates ability to identify problems, build solutions, and ship products people actually use.\\
\faGithub~\href{https://github.com/Bananz0/eGPUae}{github.com/Bananz0/eGPUae}}

\cvitem{Galaxy Book Enabler - 450+ GitHub Stars}{2024--Present\\
\textbf{PowerShell} | Registry Manipulation | Package Management\\
Another shipped product with large user base. Developed sophisticated Windows tool with hardware spoofing, automatic elevation, and smart package management through WinGet. Shows ability to understand complex systems, reverse-engineer solutions, and deliver polished end-user experiences.\\
\faGithub~\href{https://github.com/Bananz0/GalaxyBookEnabler}{github.com/Bananz0/GalaxyBookEnabler}}

\cvitem{16-Stage FIR Filter - Real-Time DSP on FPGA}{Nov 2024--Dec 2024\\
\textbf{SystemVerilog} | Altera Cyclone V | MATLAB\\
Designed and implemented complete real-time audio processing system from algorithm design through hardware testing. Created MATLAB scripts for coefficient generation, wrote SystemVerilog for FPGA implementation, and debugged hardware interfaces (SPI ADC/DAC). Achieved 19-cycle latency for 48kHz stereo audio. Demonstrates full-stack hardware development capability.\\
\faGithub~\href{https://github.com/Bananz0/16-Stage-FIR-Notch-Filter-for-Real-Time-Audio-Processing-on-FPGA}{github.com/Bananz0/16-Stage-FIR-Notch-Filter}}

\cvitem{PiBoard - Real-Time Collaborative Whiteboard}{Apr 2024--May 2024\\
\textbf{C++/Qt} | Raspberry Pi | Custom Serial Protocol\\
Built collaborative application with custom binary protocol over GPIO UART. Implemented multi-threaded architecture for <100ms latency. Directly configured hardware registers for UART control. Shows comfort working at multiple abstraction levels—from low-level hardware to application logic.\\
\faGithub~\href{https://github.com/Bananz0/PiBoard}{github.com/Bananz0/PiBoard}}

\cvitem{TSMC 65nm CPU - Complete IC Design \& Fabrication}{Sep 2023--Jun 2024\\
\textbf{Team Lead} | S-Edit, T-Spice, L-Edit, Calibre\\
Led 6-person team through complete IC design flow from concept to GDSII tape-out. Managed tight schedules, coordinated between team members, and successfully submitted for fabrication with zero violations. Achieved 15\% area reduction through iterative optimization. Demonstrates leadership and ability to deliver complex projects on deadline.}

\section{Hands-On Technical Skills}

\cvitem{Embedded Software}{C (bare metal \& RTOS), C++, ARM Cortex-M, STM32 (experience through projects), ESP32}
\cvitem{Scripting \& Automation}{Python (expert level), PowerShell (production tools), Bash scripting}
\cvitem{Linux \& Tools}{Ubuntu/Debian, systemd, Git, Docker, build systems, command-line workflows}
\cvitem{Hardware Interfaces}{SPI, I2C, UART, GPIO, ADC/DAC, PWM, interrupt handling}
\cvitem{IoT \& Protocols}{MQTT, Modbus RTU/TCP, WiFi, Bluetooth, RESTful APIs, WebSocket}
\cvitem{PCB Design}{KiCad (designed multiple boards), EAGLE, component selection, DFM principles}
\cvitem{FPGA/Digital}{SystemVerilog, Quartus Prime, Vivado, ModelSim}
\cvitem{Web/Full-Stack}{React.js, Node.js, Express.js, PostgreSQL, MongoDB (full-stack prototyping)}
\cvitem{Dev Tools}{VS Code, CLion, GDB, JTAG/SWD, oscilloscopes, logic analyzers, multimeters}

\section{Why I'm Perfect for This Startup Role}

\cvitem{I Move Fast}{\textbf{Proven track record:} Built WattsApp (full IoT system) in 3 months while managing coursework. Shipped multiple production tools used by 8,300+ people. Comfortable with rapid iteration and "good enough to ship" mentality.}

\cvitem{I Experiment \& Problem-Solve}{\textbf{Natural curiosity:} 15+ GitHub repos spanning embedded systems, automation, IoT, and infrastructure. Always building, breaking, and fixing things. Comfortable diving into unfamiliar territory and figuring it out.}

\cvitem{Full-Stack Capability}{\textbf{Versatile:} Can work across the entire stack—embedded firmware (C/C++), backend services (Python), web dashboards (React), hardware interfaces, PCB design, and DevOps. Whatever needs doing, I can learn it or build it.}

\cvitem{Team Player}{\textbf{Great communicator:} Led teams, collaborated with multidisciplinary groups, and explained technical concepts to non-technical stakeholders. Comfortable working directly with founders in tight-knit environment.}

\cvitem{Startup Mindset}{\textbf{Can-do attitude:} Don't wait for perfect specs—jump in, prototype, test, iterate. Built infrastructure serving 8+ users with 99.8\% uptime. Comfortable wearing multiple hats and doing what needs to be done.}

\section{Additional Information}
\cvitem{Location}{Based in Southampton, highly flexible for London hybrid role (2-3 days/week in office)}
\cvitem{Availability}{Graduating June 2026, available for immediate start or part-time during final semester}
\cvitem{Languages}{English (Native), Swahili (Native)}
\cvitem{Right to Work}{UK Student Visa (valid until 2026), eligible for Graduate Route visa (2-3 years work authorization)}
\cvitem{Memberships}{IEEE Member, ISACA Member}

\end{document}