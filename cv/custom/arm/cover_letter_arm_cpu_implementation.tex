\documentclass[11pt,a4paper,sans]{moderncv}
\moderncvstyle{banking}
\moderncvcolor{blue}
\renewcommand{\sfdefault}{lmss} 
\usepackage[scale=0.85]{geometry}
\usepackage{hyperref}

% Personal Information
\name{Glen}{Muthoka Mutinda}
\phone{+44 7341 625286}
\email{theglenmuthoka@gmail.com}
\social[linkedin]{glenmuthoka}
\social[github]{Bananz0}

\begin{document}

\recipient{Graduate Recruitment Team}{Arm Limited\\Cambridge, UK}
\date{\today}
\opening{Dear Hiring Manager,}
\closing{Yours sincerely,}

\makelettertitle

I am writing to apply for the Graduate Engineer position in Arm's CPU Implementation team. As a final-year Electrical \& Electronics Engineering student at the University of Southampton expecting First Class Honours or a High 2:1, I am excited by the opportunity to contribute to physical implementation of processor IP that powers billions of devices worldwide. My hands-on experience spanning the complete IC design flow from RTL to GDSII combined with my passion for optimization and low-power design I feel makes me an ideal candidate for this role.

During my second-year project, I led a six-member team through the full physical implementation lifecycle of a Digital Clock, having being mainly responsible for the Ring Oscillator for a real-time clock chip built on the TSMC 65nm process node. As Team Lead, I managed the complete design flow from circuit design through SPICE simulation, physical layout and rigorous verification using Calibre DRC/LVS. Through strategic planning and optimization, I achieved a 15\% area reduction while maintaining zero DRC violations. This experience gave me deep appreciation for the challenges of nanometre-scale design and the critical importance of power-efficiency trade-offs—challenges that directly align with the physical implementation work at Arm.

My technical foundation extends beyond IC fabrication to advanced digital design. I have designed and implemented a 16-stage FIR filter on Altera Cyclone V FPGA using SystemVerilog, achieving 19-cycle latency through careful pipeline optimization. More recently, I developed MIPSquare++, a comprehensive 5-stage pipelined MIPS processor simulator in C++ featuring intelligent hazard detection, data forwarding, and automated pipeline optimization. These projects demonstrate my ability to work across abstraction levels—from RTL microarchitecture to physical constraints—which is essential when collaborating with RTL design teams to influence implementation decisions.

I am particularly drawn to Arm's focus on methodology development and automation. In my role as Junior Software Engineer for the ARTEMIS Small Sat-1 lunar CubeSat project, I reduced system latency by 6\% through algorithm optimization and enhanced reliability by 20\% through comprehensive testing frameworks. Working within strict timing and safety requirements following NASA coding standards taught me the value of robust methodologies and automated verification—skills I am eager to apply when developing implementation flows in collaboration with EDA vendors and advancing process nodes.

My technical toolkit encompasses the hardware description languages (SystemVerilog), scripting proficiency (Python, C/C++, MATLAB), and EDA tools (Quartus Prime, Vivado, ModelSim, Cadence suite) mentioned in your requirements. Beyond coursework, I maintain 69 GitHub repositories demonstrating continuous learning and practical problem-solving—from embedded systems projects to infrastructure automation. My current final-year project on AI-driven optical authentication leverages GPU-accelerated processing with CUDA/CuPy, further strengthening my optimization mindset.

What excites me most about this position is the opportunity to work at the intersection of microarchitecture and physical design—influencing not just implementation, but the architectural decisions that define performance and efficiency. Arm's graduate programme structure, with mentoring, Grad-teach-Grads workshops, and the Global Graduate Conference, aligns perfectly with my commitment to continuous growth and collaborative engineering.

I am confident that my proven track record in physical implementation, combined with my analytical approach to complex technical challenges and enthusiasm for low-power design optimization, would make me a valuable addition to Arm's CPU Implementation team. I am eager to contribute to defining the next generation of processor IP while developing my expertise in advanced physical implementation techniques.

Thank you for considering my application. I look forward to the opportunity to discuss how my background and passion for digital hardware design can contribute to Arm's continued innovation in processor technology.

\makeletterclosing

\end{document}
